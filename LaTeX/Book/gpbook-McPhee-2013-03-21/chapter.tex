%%%%%%%%%%%%%%%%%%%% author.tex %%%%%%%%%%%%%%%%%%%%%%%%%%%%%%%%%%%
%
%%%%%%%%%%%%%%%% Springer %%%%%%%%%%%%%%%%%%%%%%%%%%%%%%%%%%

\title*{Using Graph Databases to Explore Genetic Programming Run Dynamics}
% Use \titlerunning{Short Title} for an abbreviated version of
% your contribution title if the original one is too long
\author{Nicholas Freitag McPhee, David Donatucci, and Thomas Helmuth and others(?)}
% Use \authorrunning{Short Title} for an abbreviated version of
% your contribution title if the original one is too long
\institute{Nicholas Freitag McPhee \at University of Minnesota, Morris
\and David Donatucci \at University of Minnesota, Morris
\and Thomas Helmuth \at University of Massachusetts Amherst}

\maketitle

\abstract{Each chapter should be preceded by an abstract (10--15 lines long) that summarizes the content. The abstract will appear \textit{online} at \url{www.SpringerLink.com} and be available with unrestricted access. This allows unregistered users to read the abstract as a teaser for the complete chapter. As a general rule the abstracts will not appear in the printed version of your book unless it is the style of your particular book or that of the series to which your book belongs.}

\begin{keywords}
keywords to your chapter, these words should also be indexed
\end{keywords}
\index{keywords to your chapter}
\index{these words should also be indexed}
\\

\section{Note to early reviewers}

\emph{
This is definitely still very much a work in progress, and we apologize for its incompleteness. 
We're still working through small bits of the many, many gigabytes of data that we have access to, 
looking for good stories to tell. Hopefully, however, this will give a sense of our goals, plan, 
and some of our early results. We'd certainly love feedback on which of these stories you feel are 
the most interesting, informative, and ultimately helpful for the community, as well as any 
suggestions you have regarding the telling of the stories.
}

\section{Introduction}
\label{sec:introduction}

It is common practice in most empirical evolutionary computation (EC) 
research\marginpar{It would be nice to scrape, say, the GECCO 2014 proceedings and get some 
	numbers to back this up.} to perform numerous (possibly hundreds) of runs, and then simply 
report a handful of aggregate statistics at the end that are expected to summarize and represent 
the (hopefully) complex interactions and dynamics of those many runs. Tables present values such 
as mean or median best fitnesses at the end of runs, collapsing the complexities of dozens or 
hundreds of runs into a single number, possibly with a standard deviation or (even better) a 
confidence interval to give a sense of the distribution. Often more informative are plots, which can, 
for example, show how these numbers change over time during the runs, possibly giving a sense of 
the system dynamics. These plots, however, often aggregate runs in a way that obscure important 
moments that, if explored, might reveal valuable insight into the evolutionary dynamics being 
reported.

An alternative would be to collect, store, and analyze at least some of the rich panoply of 
evolutionary and genealogical events that make up the vital low-details details of these runs. 
Databases provide a natural tool for 
storing and accessing the data, but traditional relational
databases are poorly suited for a variety of queries that are important for the genealogical analysis
we need for exploring the evolutionary dynamics of our EC runs. If, for example, we have a 
\texttt{ParentChild} table, it's easy enough to find Alice's parents. Finding Alice's grandparents,
siblings, or cousins, however, is less straightforward, with queries becomes increasingly difficult
as we move farther out across the relationship tree.

In this chapter we illustrate the use of graph databases as an alternative storage and analysis tool for
evolutionary computation runs. In \cite{donatuccianalysis} we have demonstrated that graph databases
can be an effective tool for analyzing complex genetic programming (GP) dynamics, which led directly
to a proposed change to standard sub-tree crossover in tree-based GP, \cite{mcphee:GECCO15}.
Here we will use the open source Neo4J graph database tool\footnote{\url{http://neo4j.com/}} 
to explore data from a
collection of PushGP runs on several problems drawn from a benchmark collection of introductory 
programming problems taken from \cite{helmuth:GECCO15}.

\emph{More stuff would have to be added here, including a roadmap.}

\section{Motivation}
\label{sec:motivation}

Consider the job of a paleontologist, who regularly reconstructs not just individuals but also
species and entire phylogenetic trees on the basis of handful of teeth and bones, or even just
impressions left in prehistoric mud. They rarely have DNA, so any evolutionary relationship is
inherently speculative, subject to constant debate and revision. Even with detailed DNA sequences,
the construction of phylogenetic trees for existing species is a challenge.\marginpar{A reference
	for all of this phylogenetic reconstruction stuff would be useful.}

In evolutionary computation, however, we have \emph{everything}, at least in principle. Every
selection, every mating, every mutation, and every crossover happens in our code and on our watch.
Yet we typically throw almost all that data away, reporting just aggregate statistics and summary
plots, completely failing to take advantage of our privileged position, a position most 
paleontologists would presumably eye with considerable envy. Not only does this seem an inherent
waste, these aggregations typically obscure critical moments in the dynamics of runs which might
speak volumes if explored.
\marginpar{Perhaps include a sample plot and show how it hides things? One of Tom's 
	diversity or cluster plots? A synthetic plot? Maybe that's just not necessary?}

While this sort of aggregate reporting is often valuable, allowing for important comparative
analysis of, for example, the impact of different genetic operators, it typically fails to provide
any sense of the \emph{why}. Yes, Treatment A led to better aggregate performance than 
Treatment B -- but what happened in the runs that led to that result? Any success at the end of a
run is ultimately the intricate combination of hundreds or thousands of selections, recombinations
and mutations, and if Treatment A is in some sense ``better'' than Treatment B, it must ultimately
be because it affected all those genealogical and genetic events in some significant way, biasing them
towards events that made success more likely.

Unfortunately, published research very rarely includes information that might shed light on 
these \emph{why} events. We rarely see evolved programs, for example, or any kind of post-run analysis
of those programs, and there is almost never any data or discussion of the genealogical history that
might help us understand how a successful program actually came to be. 

Sometimes this isn't included
for reasons of space and time; evolved programs, for example, are often extremely large and complex,
and a meaningful presentation and discussion of such a program could easily take up more space than
authors have available given the typical space limitations in published work.
Our suspicion, however, is that this sort of \emph{why} analysis often isn't reported because it isn't 
even \emph{done}, in no small part because it's hard. As EC researchers we're in the
``happy'' position of being able to collect anything and everything that happens in a run, but that 
leaves us with two problems: How to \emph{store} the data, and how to \emph{analyze} the data
after it's stored. Decreasing data storage costs have done much to mitigate the first problem.
If, however, one collects a very rich data set it's still easy to quickly generate terabytes of data,
and even if one has a place to put the data, one still needs reasonable tools to analyze the data.

Databases are the obvious tool for the storage task, but most common database structures and tools
don't lend themselves to the kinds of analysis that we want and need in evolutionary computation work.
Most relational and document-based databases, for example require things like complex and expensive 
recursive joins to trace significant hereditary lines, making this approach increasingly
infeasible. In exploring the dynamics of an EC run, for example, we're going to want to be able to
make connections across dozens or even hundreds of generations, which simply isn't plausible with a
relational database. (See \cite{Robinson:GraphDB:Book} for more on these 
feasability/efficiency issues.)


\section{Section Heading}
\label{sec:1}
% Always give a unique label
% and use \ref{<label>} for cross-references
% and \cite{<label>} for bibliographic references
Instead of simply listing headings of different levels we recommend to
let every heading be followed by at least a short passage of text.
Further on please use the \LaTeX\ automatism for all your
cross-references and citations. And please note that the first line of
text that follows a heading is not indented, whereas the first lines of
all subsequent paragraphs are.

\section{Section Heading}
\label{sec:2}
% Always give a unique label
% and use \ref{<label>} for cross-references
% and \cite{<label>} for bibliographic references
Instead of simply listing headings of different levels we recommend to
let every heading be followed by at least a short passage of text.
Further on please use the \LaTeX\ automatism for all your
cross-references and citations.

Please note that the first line of text that follows a heading is not indented, whereas the first 
lines of all subsequent paragraphs are.

Use the standard \verb|equation| environment to typeset your equations, e.g.
%
\begin{equation}
a \times b = c\;,
\end{equation}
%
however, for multiline equations we recommend to use the \verb|eqnarray| environment.
\begin{eqnarray}
a \times b = c \nonumber\\
\vec{a} \cdot \vec{b}=\vec{c}
\label{eq:01}
\end{eqnarray}

\subsection{Subsection Heading}
\label{subsec:2}
Instead of simply listing headings of different levels we recommend to
let every heading be followed by at least a short passage of text.
Further on please use the \LaTeX\ automatism for all your
cross-references\index{cross-references} and citations\index{citations}
as has already been described in Sect.~\ref{sec:2}.

\begin{quotation}
Please do not use quotation marks when quoting texts! Simply use the \verb|quotation| environment -- it will automatically render Springer's preferred layout.
\end{quotation}


\subsubsection{Subsubsection Heading}
Instead of simply listing headings of different levels we recommend to
let every heading be followed by at least a short passage of text.
Further on please use the \LaTeX\ automatism for all your
cross-references and citations as has already been described in
Sect.~\ref{subsec:2}, see also Fig.~\ref{fig:1}\footnote{Footnotes are easily added with this simple command.}

Please note that the first line of text that follows a heading is not indented, whereas the first lines of all subsequent paragraphs are.

% For figures use
%
\begin{figure}[b] %[b] sets the image at the bottom of the page; t = top, b = bottom, h = here%
\sidecaption
% Use the relevant command for your figure-insertion program
% to insert the figure file.
% For example, with the graphicx style use
\includegraphics[scale=.65]{figure}
%
% If no graphics program available, insert a blank space i.e. use
%\picplace{5cm}{2cm} % Give the correct figure height and width in cm
%
\caption{If the width of the figure is less than 7.8 cm use the \texttt{sidecapion} command to flush the caption on the left side of the page. If the figure is positioned at the top of the page, align the sidecaption with the top of the figure -- to achieve this you simply need to use the optional argument \texttt{[t]} with the \texttt{sidecaption} command}
\label{fig:1}       % Give a unique label
\end{figure}


\paragraph{Paragraph Heading} %
Instead of simply listing headings of different levels we recommend to
let every heading be followed by at least a short passage of text.
Further on please use the \LaTeX\ automatism for all your
cross-references and citations as has already been described in
Sect.~\ref{sec:2}.

Please note that the first line of text that follows a heading is not indented, whereas the first lines of all subsequent paragraphs are.

For typesetting numbered lists we recommend to use the \verb|enumerate| environment -- it will automatically render Springer's preferred layout.

\begin{enumerate}
\item{Livelihood and survival mobility are oftentimes coutcomes of uneven socioeconomic development.}
\begin{enumerate}
\item{Livelihood and survival mobility are oftentimes coutcomes of uneven socioeconomic development.}
\item{Livelihood and survival mobility are oftentimes coutcomes of uneven socioeconomic development.}
\end{enumerate}
\item{Livelihood and survival mobility are oftentimes coutcomes of uneven socioeconomic development.}
\end{enumerate}


\subparagraph{Subparagraph Heading} In order to avoid simply listing headings of different levels we recommend to let every heading be followed by at least a short passage of text. Use the \LaTeX\ automatism for all your cross-references and citations as has already been described in Sect.~\ref{sec:2}, see also Fig.~\ref{fig:2}.

For unnumbered list we recommend to use the \verb|itemize| environment -- it will automatically render Springer's preferred layout.

\begin{itemize}
\item{Livelihood and survival mobility are oftentimes coutcomes of uneven socioeconomic development, cf. Table~\ref{tab:1}.}
\begin{itemize}
\item{Livelihood and survival mobility are oftentimes coutcomes of uneven socioeconomic development.}
\item{Livelihood and survival mobility are oftentimes coutcomes of uneven socioeconomic development.}
\end{itemize}
\item{Livelihood and survival mobility are oftentimes coutcomes of uneven socioeconomic development.}
\end{itemize}

\begin{figure}[t] %[t] sets the image at the top of the page; t = top, b = bottom, h = here%
\sidecaption[t]
% Use the relevant command for your figure-insertion program
% to insert the figure file.
% For example, with the option graphics use
\includegraphics[scale=.65]{figure}
%
% If no graphics program available, insert a blank space i.e. use
%\picplace{5cm}{2cm} % Give the correct figure height and width in cm
%
%\caption{Please write your figure caption here}
\caption{If the width of the figure is less than 7.8 cm use the \texttt{sidecapion} command to flush the caption on the left side of the page. If the figure is positioned at the top of the page, align the sidecaption with the top of the figure -- to achieve this you simply need to use the optional argument \texttt{[t]} with the \texttt{sidecaption} command}
\label{fig:2}       % Give a unique label
\end{figure}

% Use the \index{} command to code your index words
% Make sure to inlcude the indexed word inside and outside of the brackets if you want the text to show up in your paragraph:
% e.g. This book is about \index{genetic programming}genetic programming. 
% If the text is not entered outside the brackets it will appear as: This book is about .
%
% For tables use
%
\begin{table}
\caption{Please write your table caption here}
\label{tab:1}       % Give a unique label
%
% Follow this input for your own table layout
%
\begin{tabular}{p{2cm}p{2.4cm}p{2cm}p{4.9cm}}
\hline\noalign{\smallskip}
Classes & Subclass & Length & Action Mechanism  \\
\noalign{\smallskip}\svhline\noalign{\smallskip}
Translation & mRNA$^a$  & 22 (19--25) & Translation repression, mRNA cleavage\\
Translation & mRNA cleavage & 21 & mRNA cleavage\\
Translation & mRNA  & 21--22 & mRNA cleavage\\
Translation & mRNA  & 24--26 & Histone and DNA Modification\\
\noalign{\smallskip}\hline\noalign{\smallskip}
\end{tabular}
$^a$ Table foot note (with superscript)
\end{table}
%
\section{Section Heading}
\label{sec:3}
% Always give a unique label
% and use \ref{<label>} for cross-references
% and \cite{<label>} for bibliographic references
Instead of simply listing headings of different levels we recommend to
let every heading be followed by at least a short passage of text.
Further on please use the \LaTeX\ automatism for all your
cross-references and citations as has already been described in
Sect.~\ref{sec:2}.

Please note that the first line of text that follows a heading is not indented, whereas the first lines of all subsequent paragraphs are.

If you want to list definitions or the like we recommend to use the Springer-enhanced \verb|description| environment -- it will automatically render Springer's preferred layout.

\begin{description}[Type 1]
\item[Type 1]{That addresses central themes pertainng to migration, health, and disease. In Sect.~\ref{sec:1}, Wilson discusses the role of human migration in infectious disease distributions and patterns.}
\item[Type 2]{That addresses central themes pertainng to migration, health, and disease. In Sect.~\ref{subsec:2}, Wilson discusses the role of human migration in infectious disease distributions and patterns.}
\end{description}

\subsection{Subsection Heading} %
In order to avoid simply listing headings of different levels we recommend to let every heading be followed by at least a short passage of text. Use the \LaTeX\ automatism for all your cross-references and citations citations as has already been described in Sect.~\ref{sec:2}.

Please note that the first line of text that follows a heading is not indented, whereas the first lines of all subsequent paragraphs are.

\begin{svgraybox}
If you want to emphasize complete paragraphs of texts we recommend to use the newly defined Springer class option \verb|graybox| and the newly defined environment \verb|svgraybox|. This will produce a 15 percent screened box 'behind' your text.

If you want to emphasize complete paragraphs of texts we recommend to use the newly defined Springer class option and environment \verb|svgraybox|. This will produce a 15 percent screened box 'behind' your text.
\end{svgraybox}


\subsubsection{Subsubsection Heading}
Instead of simply listing headings of different levels we recommend to
let every heading be followed by at least a short passage of text.
Further on please use the \LaTeX\ automatism for all your
cross-references and citations as has already been described in
Sect.~\ref{sec:2}.

Please note that the first line of text that follows a heading is not indented, whereas the first lines of all subsequent paragraphs are.

\begin{theorem}
Theorem text goes here.
\end{theorem}
%
% or
%
\begin{definition}
Definition text goes here.
\end{definition}

\begin{proof}
%\smartqed
Proof text goes here.
\qed
\end{proof}

\paragraph{Paragraph Heading} %
Instead of simply listing headings of different levels we recommend to
let every heading be followed by at least a short passage of text.
Further on please use the \LaTeX\ automatism for all your
cross-references and citations as has already been described in
Sect.~\ref{sec:2}.

Note that the first line of text that follows a heading is not indented, whereas the first lines of all subsequent paragraphs are.
%
% For built-in environments use
%
\begin{theorem}
Theorem text goes here.
\end{theorem}
%
\begin{definition}
Definition text goes here.
\end{definition}
%
\begin{proof}
\smartqed
Proof text goes here.
\qed
\end{proof}
%
\begin{acknowledgement}
If you want to include acknowledgments of assistance and the like at the end of an individual chapter please use the \verb|acknowledgement| environment -- it will automatically render Springer's preferred layout.
\end{acknowledgement}

\bibliographystyle{spbasic}
\bibliography{gp-bibliography,chapter}
